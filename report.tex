\documentclass{article}
\begin{document}

\begin{titlepage}
\vspace*{\stretch{1.0}}
\begin{center}
    \Large\textbf{Routing using Quagga: An example with OSPF}\\
    \textit{by}\\
    \large\textit{Abhishek Balaji \\ Adithya K}
\end{center}
\vspace*{\stretch{2.0}}
\end{titlepage}
\tableofcontents
\clearpage
\section{Introduction}
\pagenumbering{arabic}
Network Routing refers to the action of an electronic network in transferring data from one part of the network to another in an
efficient and an optimized way. The Network typically consists of a bunch of forwarding elements (Hardware) which perform the 
actual transmission, receiving of data and a software sitting on top of them, which co-ordinate and take decisions on
which next hardware element the piece of data should be given to. Quagga is one such software, a Network Routing Software.
It runs on Unix-like systems (GNU/Linux, Solaris, FreeBSD, NetBSD). Quagga is a Software Suite, having implementations of the famous 
routing protocols of the Internet viz. Routing Information Protocol (RIP), Open Shortest Path First (OSPF), Border Gateway Protocol (BGP)
and also support for BABEL, OLSR (Wireless Mesh Routing Protocols).\\
Talk about Zebra, history etc. here

\subsection{Architecture}
Describe in detail the architecture of quagga, zebra, zserv API, the various daemons
\subsection{Usecases}
Mention specifically who currently use quagga and for what purpose.
\clearpage
\section{OSPF}
Do we need this???
\clearpage
\section{Testbed Description}
This section will describe our experiment, what we intend to do
\subsection{Experiment 1}
Set up of OSPF showing how the routing table is created
\subsection{Experiment 2}
Tear down a link and display how packets are re-routed along with the new routing table
\clearpage
\section{Performance Measurement Results}
This section will display the ping graphics and the traceroute graphics for both experiment 1 and experiment 2
\subsection{Experiment 1}
\subsection{Experiement 2}
\clearpage
\section{Conclusions and Suggestions}
This section needs to talk about what use-case we solved by using quagga and what we learnt using Quagga
Also, if we have any feedback for this assignment we can give it here.
\end{document}
